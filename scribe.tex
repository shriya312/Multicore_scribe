%
% This is the LaTeX template file for lecture notes for EE 382C/EE 361C.
%
% To familiarize yourself with this template, the body contains
% some examples of its use.  Look them over.  Then you can
% run LaTeX on this file.  After you have LaTeXed this file then
% you can look over the result either by printing it out with
% dvips or using xdvi.
%
% This template is based on the template for Prof. Sinclair's CS 270.

\documentclass[twoside]{article}
\usepackage{graphics}
\usepackage{cancel}
\usepackage{centernot}
\usepackage{amsmath}
 \usepackage[margin=1in]{geometry} 
 \usepackage{graphicx}
\setlength{\oddsidemargin}{0.25 in}
\setlength{\evensidemargin}{-0.25 in}
\setlength{\topmargin}{-0.6 in}
\setlength{\textwidth}{6.5 in}
\setlength{\textheight}{8.5 in}
\setlength{\headsep}{0.75 in}
\setlength{\parindent}{0 in}
\setlength{\parskip}{0.1 in}

%
% The following commands set up the lecnum (lecture number)
% counter and make various numbering schemes work relative
% to the lecture number.
%
\newcounter{lecnum}
\renewcommand{\thepage}{\thelecnum-\arabic{page}}
\renewcommand{\thesection}{\thelecnum.\arabic{section}}
\renewcommand{\theequation}{\thelecnum.\arabic{equation}}
\renewcommand{\thefigure}{\thelecnum.\arabic{figure}}
\renewcommand{\thetable}{\thelecnum.\arabic{table}}

%
% The following macro is used to generate the header.
%
\newcommand{\lecture}[4]{
   \pagestyle{myheadings}
   \thispagestyle{plain}
   \newpage
   \setcounter{lecnum}{#1}
   \setcounter{page}{1}
   \noindent
   \begin{center}
   \framebox{
      \vbox{\vspace{2mm}
    \hbox to 6.28in { {\bf EE 382C/361C: Multicore Computing
                        \hfill Fall 2016} }
       \vspace{4mm}
       \hbox to 6.28in { {\Large \hfill Lecture #1: #2  \hfill} }
       \vspace{2mm}
       \hbox to 6.28in { {\it Lecturer: #3 \hfill Scribe: #4} }
      \vspace{2mm}}
   }
   \end{center}
   \markboth{Lecture #1: #2}{Lecture #1: #2}
   %{\bf Disclaimer}: {\it These notes have not been subjected to the
   %usual scrutiny reserved for formal publications.  They may be distributed
   %outside this class only with the permission of the Instructor.}
   \vspace*{4mm}
}

%
% Convention for citations is authors' initials followed by the year.
% For example, to cite a paper by Leighton and Maggs you would type
% \cite{LM89}, and to cite a paper by Strassen you would type \cite{S69}.
% (To avoid bibliography problems, for now we redefine the \cite command.)
% Also commands that create a suitable format for the reference list.
\renewcommand{\cite}[1]{[#1]}
\def\beginrefs{\begin{list}%
        {[\arabic{equation}]}{\usecounter{equation}
         \setlength{\leftmargin}{2.0truecm}\setlength{\labelsep}{0.4truecm}%
         \setlength{\labelwidth}{1.6truecm}}}
\def\endrefs{\end{list}}
\def\bibentry#1{\item[\hbox{[#1]}]}

%Use this command for a figure; it puts a figure in wherever you want it.
%usage: \fig{NUMBER}{SPACE-IN-INCHES}{CAPTION}
\newcommand{\fig}[3]{
			\vspace{#2}
			\begin{center}
			Figure \thelecnum.#1:~#3
			\end{center}
	}
% Use these for theorems, lemmas, proofs, etc.
\newtheorem{theorem}{Theorem}[lecnum]
\newtheorem{lemma}[theorem]{Lemma}
\newtheorem{proposition}[theorem]{Proposition}
\newtheorem{claim}[theorem]{Claim}
\newtheorem{corollary}[theorem]{Corollary}
\newtheorem{definition}[theorem]{Definition}
\newenvironment{proof}{{\bf Proof:}}{\hfill\rule{2mm}{2mm}}

% **** IF YOU WANT TO DEFINE ADDITIONAL MACROS FOR YOURSELF, PUT THEM HERE:

\begin{document}
%FILL IN THE RIGHT INFO.
%\lecture{**LECTURE-NUMBER**}{**DATE**}{**LECTURER**}{**SCRIBE**}
\lecture{10}{September 27}{Vijay Garg}{Shriya Nair}
%\footnotetext{These notes are partially based on those of Nigel Mansell.}

% **** YOUR NOTES GO HERE:

% Some general latex examples and examples making use of the
% macros follow.  
%**** IN GENERAL, BE BRIEF. LONG SCRIBE NOTES, NO MATTER HOW WELL WRITTEN,
%**** ARE NEVER READ BY ANYBODY.
\section{Puzzle: Sort 2 arrays}
\textbf{Description}:\\
Two sorted A and B of n elements each have to be merged into a sorted array C\\\\
\textbf{Solution 1}: Sequential Algorithm\\
-Time = O(n)\\
-Work = O(n)\\
-using two pointers \\\\
\textbf{Solution 2}: Parallel Algorithm\\
- select an element, find its rank (index + position from binary search)\\
- Time = $O(\log n)$ \\
- Work = $O(n\log n)$\\\\
\textbf{Solution 3}: Cascaded Algorithm - the Work-Optimal Solution\\
-Divide n array into n/log n groups each of size log n.\\
	Number of splitters = $(n/\log n)$\\ 
	Time taken to search for a splitter = $O (\log n)$\\
    Arrows will never cross because both arrays are sorted\\

\includegraphics[scale=0.58]{scribefor.JPG}


- Fill in only splitters \\
- Find sublists between and such that they are between Q and A \\
Max number of sublists = $O(n/\log n)$\\
Size of each list = $\log n$\\
Two lists of logn size can be merged in $O(\log n)$\\

\begin{table}[ht]
\begin{tabular} {c c c}
step & T(n) & W(n) \\
%heading
\hline
Rank & $O(\log n)$ & O(n) \\
Merging & $O(\log n)$ &$O(n / \log n . \log n)$ = O(n)\\
\textbf{total:} & $O(\log n)$ & O(n) \\
\hline
\end{tabular}
\end{table}

\section{New Puzzle}

\textbf{Problem}: Parallel Prefix sum\\
\textbf{Description}: Consider an array on size n not necessarily sorted. Compute the recurring sum. (also called scan)\\
\textbf{Example}: 1,2,3,4,5,6 \\ 
Output : 1,3,6,10,15,21\\
\\

\textbf{Solution 1}: Sequential Algorithm \\
T(n) = O(n) \\
W(n) = O(n) \\
  
\section{Consistency Conditions}
- cannot use locks on datastructures- expensive, sequentializing.\\
- limiting parallelization of code. \\
Example: 
\begin{verbatim}
---push(40)----  ----pop(?)----
      ----push(30)----
\end{verbatim}
When an element is popped will it return 40 or 30? \\ 
- Need a definition for consistency.
\subsection{Sequential Consistency}
- Defined by Lamport\\
Consider a method foo:
\begin{verbatim}
	s.foo(arg1, arg2)
\end{verbatim}
where.. \\
1. s is the object\\
2. foo: name of the method \\
3. the statement is the invocation\\
4. arg1, arg2 are the arguments\\
5. s.response is the return value. Possible values: OK, value, exception \\\\
Consider two operations e,f. \\
Conventions:\\
inv(e) : invocation of e\\
resp(e): response generated after invoking e\\
proc(e) : process e is on\\
\\
\textbf{History of operations}: (H, \textless\textsubscript{H}) where \textless\textsubscript{H} is the real-time order. \\
\textbf{Definition}: e \textless\textsubscript{H} f = resp(e) occurred before inv(f)\\
Example: 
\begin{verbatim}
---push(40)---- 
    (e)
                   ----push(30)----
                         (f)
\end{verbatim}

Consider e, f :
\begin{verbatim}
---push(40)---
      (e)
            ---push(30)---
                (f)
\end{verbatim}
e \cancel{\textless}\textsubscript{H} f \&\& f \cancel{\textless}\textsubscript{H} e $\implies$ e is concurrent with f\\

Properties of the relation \textless\textsubscript{H}:\\
1. Irreflexive : e \textless\textsubscript{H} f $\centernot\implies$ f \cancel{\textless}\textsubscript{H} e\\
2. Transitive : e \textless\textsubscript{H} f \&\& f \textless\textsubscript{H} g $\implies$ e \textless\textsubscript{H} g \\
$\implies$ Asymmetric\\
Hence, (H,\textless\textsubscript{H}) is a partially ordered set or poset

\subsection{Process Order}
e \textless\textsubscript{H} f are in process order iff:\\
proc (e) = proc (f) and resp(e) occurred before inv(f) \\

\textless\textsubscript{po} $\subseteq$ \textless\textsubscript{H} \\
e \textless\textsubscript{po} f $\implies$  e \textless\textsubscript{H} f but the reverse may not be true

\subsection{Sequential History}
A history (H, \textless\textsubscript{H}) is sequential if \textless\textsubscript{H} is a total order. \\
A poset (X, \textless) is a total order if $\forall$ distinct (x,y) belongs to X, (x \textless\textsubscript{H} y) or ( y \textless\textsubscript{H} x) \\
 
\textbf{Legal Sequential History:} \\
A sequential history S is legal if it satisfies sequential specifications of the objects. \\


\section*{References}
\beginrefs
\bibentry{1}{\sc Vijay K Garg},
Introduction to Multicore Computing

\endrefs


\end{document}





